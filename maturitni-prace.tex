% Šablona pro maturitní práci Gymnázia Jírovcova 8, České Budějovice
% Autoři šablony: Jonáš Havelka, Michal Kočer, Daniel Sýkora
% Typ dokumentu: report
% veškeré úpravy v soubor MP.sty (styl maturitní práce)
\documentclass[12pt]{report}
% %%%%%%%%%%%%%%%%%%%%%%%%%%%%%%%%%%%%%%%%%%%%%%%%%%%%%%%
\usepackage{MP}						  % Import stylu maturitní práce
\author{Kryštof Maxera}                  % AUTOR PRÁCE
\title{Konstrukce dronu}    % NÁZEV PRÁCE
\date{14. února 2025}                 % DATUM ODEVZDÁNÍ PRÁCE
\vedouci{Dr.rer.nat Michal Kočer} % VEDOUCÍ PRÁCE
\place{V Českých Budějovicích}
\skolnirok{2024/2025}                  % ŠKOLNí ROK
\logo{\includegraphics[scale=1.25]{GJ8_logotyp}} %Logo školy
%%%%%%%%%%%%%%%%%%%%%%%%%%%%%%%%%%%%%%%%%%%%%%%%%%%%%%%%%%%%%%%%%%%
\begin{document} %%%%%%% začátek dokumentu
%%%%%%%%%%%%%%%%%%%%%%%%%%%%  Titulní stránka + úvodní povinné stránky
\pagenumbering{roman}                   % číslování stránek římskými číslicemi
	\mytitlepage						% Vygenerování titulní strany
	
	\prohlaseni{
		Prohlašuji, že jsem tuto práci vypracoval samostatně s vyznačením všech použitých pramenů.
	}	
	
	\abstrakt{
		\lipsum[1]						% Abstrakt 
	}{
		\lipsum[1]						% Klíčová slova
	}
	
	\podekovani{
		\lipsum[2]						% Poděkování
	}
	
   {\tableofcontents\newpage}			% Obsah
	
%%%%%%%%%%%%%%%%%%%%%%%%%%%% VLASTNÍ PRÁCE
\addtocounter{page}{1}		% Posunutí countru stránek
\pagenumbering{arabic}		% Číslování stránek arabskými číslicemi
\chapter*{Úvod}     % úvod práce 
	
\lipsum[1]	
	
%%%%%%%%%%%%%% TEORETICKÁ ČÁST %%%%%%%%%%%%%%%%%%	
\part{Úvod do světa dronů}  % název teoretické části (nenechávejte Teoretická část)
	
\chapter{Co to je dron?}
			
\section{Druhy dronů}
	Odkaz v závorkách: \parencite[see][page 900]{einstein}\\
	Odkaz: \cite{knuthwebsite}\\
	A odkaz pod čarou: \footcite[see][s. 42]{latexcompanion}\\
	Dobrý den, ahoj, \gls{atd}\\
	Praha, \gls{tj} hlavní město ČR
	
	\begin{table}
 		\caption{Testovací tabulka}
		\label{tab:test2}
			\begin{tabular}{ccccc}
				1 & 1 & 1  & 1  & 1  \\
				1 & 2 & 3  & 4  & 5  \\
				1 & 3 & 6  & 10 & 15 \\
				1 & 4 & 10 & 30 & 45
				\end{tabular}
	\end{table}

\lipsum[2]

\chapter[Stručná historie dronů]{Stručná historie dronů}
%%% v obsahu se objeví jen to co je v hranatých závorkách
\begin{figure}
  \includegraphics[width=\linewidth]{test.jpg}
  \caption{Testovací}
  \label{fig:test}
\end{figure}
Obrázek \ref{fig:test} ukazuje Shangai z Pixabay.\\
Tabulka \ref{tab:test2} ukazuje hádejte, co.
	
\lipsum[3]

\chapter{Anatomie dronu}
			
\section{Motory}

\lipsum[1]

\section{ESP}

\lipsum[2]

\section{Baterie}

\lipsum[3]

\section{Raspberry Pi Pico}

\lipsum[4]

\chapter{Fyzika letu dronu}

\lipsum[1]

%%%%%%%%%%%%%% PRAKTICKÁ ČÁST %%%%%%%%%%%%%%%%%%	
\part{Konstrukce kvadrokoptéry} % název praktické části (nenechávejte název Praktická část)

\chapter{Součástky}
\lipsum[1]	

\chapter{Průběh Konstrukce}

\lipsum[1]	

Výpis programu \nameref{lst:hello_world}  naleznete ve výpise \ref{lst:hello_world}.

\begin{lstlisting}[title={Program hello.c}, caption={hello.c}, label={lst:hello_world}]
#include <stdio.h>
#define CISLO 10

int main(void) {
	int i = CISLO;

	print("Hello World!\n");
	print("%d", i);

	return (0);
}
\end{lstlisting}

\lipsum[1]	

\begin{lstlisting}[numbers=none, title={Příklad výstupního souboru}]
11.0524
5.5954
6.7996
13.8584
15.1357
Soucet: 52.4415
\end{lstlisting}

\chapter{Program}

\lipsum[1]

\chapter{Schéma zapojení}

\lipsum[1]

%%%%%%%%%%%%% ZÁVĚR
\chapter*{Závěr}
	
\lipsum[1]
	
\nocite{*}
\printbibliography					% Vytvoří seznam literatury
\addcontentsline{toc}{chapter}{Bibliografie}
\printglossary[title={Zkratky}]		% Vytvoří seznam zkratek
\listoffigures						% Vytvoří seznam obrázků
\listoftables						% Vytvoří seznam tabulek

%%%%%%%%%%%%% PŘÍLOHY - APPENDIX 	
\begin{appendices}
	\chapter{Fotografie zkonstruované kvadrokoptéry}	
	\lipsum[1]
    	%\pitem{Fotky z pokusů}
    	%\eitem{Vlastní program}
    	%\eitem{Dokumentace}
    	%\eitem{Testovací data}
	\chapter{Kód programu Raspberry Pi Pico}  
\end{appendices}
%%%%%%%%%%%%%%%
\end{document}
%%%%%%%%%%%%%%%%%%%% KONEC %%%%%%%%%%%%%%%%%%%%%%%%%
% Šablona pro maturitní práci Gymnázia Jírovcova 8, České Budějovice
% Autoři šablony: Jonáš Havelka, Michal Kočer, Daniel Sýkora
% Typ dokumentu: report
% veškeré úpravy v soubor MP.sty (styl maturitní práce)
\documentclass[12pt]{report}
% %%%%%%%%%%%%%%%%%%%%%%%%%%%%%%%%%%%%%%%%%%%%%%%%%%%%%%%
\usepackage{listingsutf8}
\usepackage{float}
\usepackage{MP}
						  % Import stylu maturitní práce
\author{Kryštof Maxera}                  % AUTOR PRÁCE
\title{Konstrukce dronu}    % NÁZEV PRÁCE
\date{14. února 2025}                 % DATUM ODEVZDÁNÍ PRÁCE
\vedouci{Dr.rer.nat Michal Kočer} % VEDOUCÍ PRÁCE
\place{V Českých Budějovicích}
\skolnirok{2024/2025}                  % ŠKOLNí ROK
\logo{\includegraphics[scale=1.25]{GJ8_logotyp}} %Logo školy
%%%%%%%%%%%%%%%%%%%%%%%%%%%%%%%%%%%%%%%%%%%%%%%%%%%%%%%%%%%%%%%%%%%
\begin{document} %%%%%%% začátek dokumentu
%%%%%%%%%%%%%%%%%%%%%%%%%%%%  Titulní stránka + úvodní povinné stránky
\pagenumbering{roman}                   % číslování stránek římskými číslicemi
	\mytitlepage						% Vygenerování titulní strany
	
	\prohlaseni{
		Prohlašuji, že jsem tuto práci vypracoval samostatně s vyznačením všech použitých pramenů.
	}	
	
	\abstrakt{
		\lipsum[1]						% Abstrakt 
	}{
		\lipsum[1]						% Klíčová slova
	}
	
	\podekovani{
		\lipsum[2]						% Poděkování
	}
	
   {\tableofcontents\newpage}			% Obsah
	
%%%%%%%%%%%%%%%%%%%%%%%%%%%% VLASTNÍ PRÁCE
\addtocounter{page}{1}		% Posunutí countru stránek
\pagenumbering{arabic}		% Číslování stránek arabskými číslicemi
\chapter*{Úvod}     % úvod práce 
	
\lipsum[1]	
	
%%%%%%%%%%%%%% TEORETICKÁ ČÁST %%%%%%%%%%%%%%%%%%	
\part{Úvod do světa dronů}  % název teoretické části (nenechávejte Teoretická část)
	
\chapter[Definice a charakteristika dronů]{Definice a charakteristika dronů}

Dron je definován jako zařízení nebo stroj schopný vykonávat úkoly bez nutnosti přímé fyzické přítomnosti člověka. Tato zařízení lze rozdělit do dvou základních kategorií.

První kategorii tvoří plně autonomní roboti, u jichž je přítomnost člověka vyžadována primárně z kontrolních a bezpečnostních důvodu. Pilot nebo operátor zde většinou nezasahuje do aktivního řízení, ale v případě potřeby může převzít kontrolu. Typickým příkladem jsou autonomní bezpilotní letadla s možností vzdáleného ovládání nebo samořízené motorové vozidlo, které ke svému provozu nepotřebuje řidiče přítomného ve vozidle.

Druhá kategorie je pro veřejnost známější. Ta je také nazývána drony, přestože její součástí jsou dálkově ovládaná zařízení, která nejsou plně autonomní. Do této skupiny patří široce známé kvadrokoptéry a další multikoptéry, stejně jako autíčka na dálkové ovládání.

Důvodem časté záměny těchto dvou kategorií je překrývání některých funkcí, neboť i dálkově ovládané kvadrokoptéry využívají automatické systémy, například pro samovyvažování, které jsou nezbytné pro jejich stabilní let.\\

Drony lze obecně rozdělit do několika hlavních skupin na základě prostředí, ve kterém operují:
\begin{itemize}
	\item \textbf{Bezpilotní letadla} (UAVs - Unmanned Aerial Vehicles)
	\item \textbf{Bezpilotní pozemní vozidla} (UGV - Unmanned Ground Vehicle)
	\item \textbf{Hladinové plavidla bez posádky} (USV - Unmanned Surface Vehicle)
	\item \textbf{Dálkově ovládané podvodní vozidla} (ROUV - Remotely Operated Underwater Vehicles)
	\item \textbf{Bezpilotní kosmické lodě} (Uncrewed spacecraft)
\end{itemize}

Přestože označení dron lze použít pro širokou škálu zařízeních, pro širokou veřejnost je toto slovo primárně spjaté s dálkově ovládanými bezpilotními letadly. Samo o sobě však toto označení není chybné. V této práci se zaměřujeme na konstrukci kvadrokoptéry, která spadá do kategorie bezpilotních letadel. O té práce pojednává podrobněji. \cite{mainbook}

\section{Bezpilotní letadla}

Bezpilotní letadlo je definováno jako zařízení určené k provozu ve vzdušném prostředí, které je buď řízeno dálkově operátorem, nebo schopno autonomního letu díky integrovanému softwaru a palubním senzorům. 

Tato zařízení využívají pokročilé technologie pro navigaci, stabilizaci, komunikaci a sběr dat, přičemž jejich provozní komponenty se liší v závislosti na specifickém účelu použití. Obecně však tato zařízení zahrnují senzory nezbytné pro stabilizaci letu, jako je gyroskop a akcelerometr, spolu se senzory či moduly umožňujícími komunikaci. Komunikační technologie obvykle zahrnují přenos dat prostřednictvím rádiových vln, Wi-Fi, nebo mobilních sítí.

Tato zařízení lze dále klasifikovat na základě specifických parametrů, jako je typ konstrukce křídla, hmotnost, zdroj napájení či funkční zaměření.\\

Při klasifikaci na základě typu konstrukce křídla lze bezpilotní letadla rozdělit do dvou hlavních skupin:
\begin{itemize}
	\item Rotorová letadla - zahrnující jednorotorové a vícerotorové varianty, jako jsou trikoptéry, kvadrokoptéry, hexakoptéry a oktokoptéry.
	\item Letadla s pevnými křídly - zahrnují drony, které vyžadují pohyb vpřed k generování vztlaku pomocí křídel. Patří sem také hybridní drony s vertikálním vzletem a přistáním, jež nevyžadují přistávací dráhu.
\end{itemize}

Zařazení na základě zdroje napájení:
\begin{itemize}
	\item Bateriový pohon - nejčastěji lithium-polymerové (Li-Po) nebo lithium-iontové (Li-Ion) baterie
	\item Benzínový pohon - spalovací motor poháněný benzínem
	\item Vodíkový pohon - napájení je zajištěno vodíkovými palivovými články, které generují elektrickou energii chemickou reakcí
	\item Solární pohon - solární panely umístěné na povrchu dronu zajišťují nepřetržité nabíjení během letu
\end{itemize}

Zařazení na základě nejběžnějších funkčních kategorií:
\begin{itemize}
	\item rekreační využití
	\item letecká fotografie a videografie
	\item pátrací a záchranné operace
	\item vojenský průmysl
	\item stavební průmysl, monitorování a měření
	\item zemědělství
	\item dopravní a logistické služby
\end{itemize}

Tato skupina zařízení byla po dlouhou dobu spojována především s vojenským průmyslem. V současnosti však díky široké škále aplikací nacházejí stále větší uplatnění i v civilních oblastech, jako je průmysl, zemědělství nebo vědecký výzkum. Díky klesajícím cenám se osobní kvadrokoptéry stávají stále populárnějšími také pro rekreační účely. \cite{mainbook} \cite{whatisadrone}\\

\section{Bezpilotní pozemní vozidla}
Jedná se o pozemní vozidla bez potřebné fyzické přítomnosti člověka. Ve srovnání se vzdušnými bezpilotními drony jsou mnohem jednodušší na konstrukci, protože nevyžadují překonávání fyzikálních zákonů spojených s letem. Tato vlastnost přispívá k jejich širokému využití napříč různými sektory. Lze je nalézt například v zemědělství jako samosklízecí traktory, v oblasti samořídících dopravních prostředků, v těžebním průmyslu, v automatizovaných skladech s roboty pro transport zboží nebo v úklidovém sektoru, kde se využívají autonomní vysavače. Své uplatnění nacházejí také ve vojenském sektoru. \cite{mainbook}

\section{Hladinové plavidla bez posádky}
Plavidla pohybující se po mořské nebo sladkovodní hladině. Často jsou využívána pro těžební operace na moři, vědecký výzkum či monitorování vodních ekosystémů. Nelze opomenout jejich vojenské využití. Tato zařízení používají pro komunikaci obdobné technologie jako bezpilotní letadla. \cite{mainbook}

\section{Dálkově ovládané podvodní vozidla}
Podvodní zařízení, určená převážně pro průzkumné a vědecké účely, představují klíčový nástroj pro studium mořského prostředí. Jejich fungování se však výrazně liší od ostatních autonomních systémů, a to kvůli technickým výzvám spojených s provozem ve velkých hloubkách. V těchto podmínkách je použití rádiových vln pro komunikaci téměř nemožné kvůli jejich omezené prostupnosti vodou. Namísto bezdrátové komunikace jsou proto tato zařízení často spojena s mateřskou lodí nebo ponorkou pomocí robustního kabelu, který slouží nejen jako přenosové médium pro data, ale i jako zdroj energie. \cite{mainbook}

\section{Bezpilotní kosmické lodě}
Vesmír představuje ideální prostředí pro využití dronů, neboť je pro lidskou přítomnost extrémně nehostinný. Vesmírné drony nabízejí méně rizikové řešení pro dosažení různých cílů bez nutnosti návratu zařízení zpět na Zemi. Jejich využití zahrnuje kosmický prostor, kde slouží k prozkoumávání vzdálených objektů, povrchy nehostinných planet, kde fungují jako rovery, a oběžnou dráhu Země, kde podporují fungování klíčových technologií, jako je například GPS. \cite{mainbook}\\

Využití všech těchto typů dronů má společnou vlastnost. Jejich využití je na místech, kde lze pracovní sílu člověka nahradit automatickým zařízením nebo místech, které jsou příliš riskantní pro přítomnost člověka. Jejich nasazení se neustále rozšiřuje a lze bez pochybností předpokládat, že v budoucnu bude jejich význam dále narůstat.

\chapter[Anatomie multikoptéry]{Anatomie multikoptéry}
Každá multikoptéra, ať už profesionálně vyrobená nebo sestavená v domácích podmínkách, se liší svým konstrukčním hardwarem. Nicméně většinu těchto zařízení spojuje použití podobných klíčových komponent. Tato kapitola se zaměřuje na podrobný popis nejběžnějších součástek, které byly využity při konstrukci kvadrokoptéry v praktické části práce.

\section{Rám}
Rám dronu tvoří základní konstrukční prvek, který drží všechny komponenty pohromadě. Typ rámu použitý u dronu má zásadní vliv na jeho celkové vlastnosti. Klíčovým aspektem při výběru a návrhu rámu je nalezení optimálního poměru mezi hmotností, velikostí a pevností, aby konstrukce byla co nejvíce odolná vůči nárazům a pádům, zároveň však nezvyšovala zbytečně hmotnost zařízení. Tento vyvážený design hraje významnou roli ve stabilitě, ovladatelnosti a životnosti dronu.

\subsection{Tvar}
První klíčovou vlastností rámu, na kterou je třeba se zaměřit, je jeho tvar. Konstrukce rámu musí odpovídat počtu vrtulí, což vyžaduje adekvátní počet ramen. Například trikoptéry disponují třemi rameny, hexakoptéry šesti, zatímco nejběžnější kvadrokoptéry mají čtyři ramena.Důležitým konstrukčním aspektem je také úhel mezi jednotlivými rameny, který ovlivňuje nejen stabilitu a letové vlastnosti dronu, ale také případné zorné pole připevněné kamery.

\subsection{Materiál}
Výběr materiálu rámu výrazně ovlivňuje dvě klíčové vlastnosti dronu: hmotnost a odolnost. Existuje široká škála materiálů vhodných pro konstrukci rámu. Základním kritériem je dostatečná pevnost materiálu, která umožňuje jeho použití jako nosné struktury. Některé materiály poskytují výrazné výhody, díky nimž jsou preferovány.\\

Nejčastěji používané materiály zahrnují:
\begin{itemize}
	\item Uhlíkové vlákno - vynikající pevnost a nízká hmotnost, vysoká cena, pro profesionální konstrukce
	\item Dřevo - snadná dostupnost, výborná tvarovatelnost, vhodné pro malé multikoptéry
	\item Hliník - vysoká pevnost a odolnost, nízká cena, vysoká hmotnost
	\item Pěna - převážně u malých dronů díky své nízké hmotnosti, omezená stabilita
	\item Plast - levný, lehký a snadno tvarovatelný materiál, možné použití 3D tisku
\end{itemize}

\subsection{Velikost}
V neposlední řadě je potřeba zvolit správnou velikost dronu.  Obecně platí, že s rostoucí velikostí dronu se zvyšuje jeho hmotnost, ale zároveň i stabilita. Volba optimální velikosti závisí na požadovaných letových vlastnostech a plánovaném zatížení dronu. \cite{mainbook} \cite{dojo} \cite{ultimateguide}

\section{Motory}
Existuje mnoho různých variant ze kterých vybírat. Každá se však svým provedením liší.\\
Motor se skládá ze dvou částí, rotoru a statoru. Rotor je ta část motoru, která při používání rotoru točí a stator je ta část , která vždy stojí na jednom místě.


\subsection{Inrunner a outrunner}
Motory lze rozdělit do dvou podkategorií: inrunner a outrunner, přičemž klíčový rozdíl spočívá v uspořádání rotoru a statoru.\\
\textbf{Inrunner motory} mají rotor uvnitř statoru, otáčí se pouze hřídel motoru a vnější plášť zůstává statický. Tyto motory se vyznačují vysokými otáčkami (vyšší kV) a nízkým točivým momentem, což je ideální pro situace vyžadující rychlost.\\
\textbf{Outrunner motory} mají rotor na vnější straně statoru, takže se otáčí celý plášť motoru. Tento design poskytuje nižší otáčky (nižší kV), ale vyšší točivý moment, což je ideální pro přímý pohon bez převodovky, například u vrtulí dronů nebo elektrokol. 


\subsection{Kartáčové a bezkartáčové motory}
Jedná se o dvě hlavní konstrukce motorů, lišící se způsobem přenosu energie a účinností.\\
\textbf{Kartáčové motory} používají mechanické kartáče k přenosu elektrické energie na rotor, kde jsou umístěny cívky. Stator je tvořen permanentními magnety. Proud procházející cívkami vytvoří magnetické pole, které odpuzuje celý rotor tak, aby generoval točivý moment. Tento design je jednoduchý a levný, ale méně účinný kvůli tření kartáčů, což způsobuje vyšší opotřebení a ztráty energie.\\
\textbf{Bezkartáčové motory} mají cívky na statoru a rotor tvořený permanentními magnety. Energie je přenášena elektronicky pomocí regulátoru otáček (ESC), který ve správný čas mění polaritu cívky tak, aby odpuzovala permanentní magnety na rotoru. Tento způsob eliminuje tření kartáčů. Bezkartáčové motory jsou účinnější, spolehlivější a vhodné pro aplikace vyžadující vysoký výkon a přesnost, například drony nebo moderní elektromobily.

\subsection{AC a DC}
V neposlední řadě je klíčové správné spárování baterie s motorem. Baterie obvykle využívají stejnosměrný proud (DC), zatímco bezkartáčové motory povětšinou vyžadují střídavý proud (AC). V takových případech je nezbytné zvolit vhodný regulátor otáček (ESC), který zajistí správnou transformaci a distribuci elektrického proudu odpovídající požadavkům motoru.  Kartáčové motory povětšinou ke svojí funkci vyžadují stejnosměrný proud. \cite{mainbook} \cite{dojo} \cite{ultimateguide} \cite{motors}

\section{Vrtule}

Vrtule fungují podobně jako šroub. Jejich lopatky při otáčení “zavrtávají” do vzduchu a vytvářejí sílu, která pohání objekt vpřed nebo vzad.\\
Existují dva hlavní typy vrtulí: tažné (regular propellers) a tlačné (pusher propellers)\\
Tažné vrtule generují tah směrem dopředu a jsou navrženy tak, aby se otáčely po směru hodinových ručiček (CW). Naproti tomu tlačné vrtule vytvářejí sílu tlačící systém vpřed a otáčejí se proti směru hodinových ručiček (CCW). Kombinace těchto dvou typů vrtulí zajišťuje vyvážený, stabilní a snadno ovladatelný systém. \cite{mainbook} \cite{dojo}

\section{Flight controller}

Často se označuje jako jeho „mozek“ celého dronu. Hlavní funkcí flight controlleru je automatické vyvažování dronu. Vyrovnává změny v pohybu, naklánění a rotaci, aby dron zůstal stabilní i v obtížných podmínkách, jako je vítr nebo nerovnoměrné rozložení hmotnosti. Díky tomu se pilot nemusí soustředit na neustálé drobné korekce, což usnadňuje manévrování.\\

Nejběžnější senzory součástí flight controlleru:
\begin{itemize}
	\item Akcelerometr -  Měří zrychlení ve třech osách (x, y, z) a pomáhá určit směr gravitace, což je nezbytné pro vyhodnocení naklonění dronu.
	\item Gyroskop - Sleduje rotaci a úhlové změny dronu, což umožňuje rychlé a přesné úpravy pro udržení stability.
\end{itemize}

Flight controller přijímá data ze senzorů a na jejich základě upravuje výkon jednotlivých motorů. \cite{mainbook}

\section{Komunikační systém}
Komunikační systém dronu slouží k přenosu informací mezi pilotem a samotným zařízením. Základem je RC ovladač, který se skládá z vysílače (drženého pilotem) a přijímače umístěného na dronu. Přijímač je přímo propojen s řídicí jednotkou (Flight Controllerem), která interpretuje přijaté signály a převádí je na odpovídající povely pro motory a další komponenty.\\
Pro základní ovládání jsou využívány minimálně čtyři komunikační kanály, které odpovídají následujícím funkcím: náklon do stran (Roll), náklon vpřed a vzad (Pitch), otáčení kolem svislé osy (Yaw) a regulace výkonu motorů (Throttle). Alternativně může být pro komunikaci využita technologie Bluetooth, která umožňuje ovládání dronu prostřednictvím mobilních zařízení.

\section{Baterie}

Baterie je zdrojem energie pro celý dron. Nejčastěji používané jsou lithiumpolymerové (LiPo) a lithium-iontové (Li-ion) baterie díky jejich vysoké energetické hustotě, nízké hmotnosti a schopnosti dodávat vysoký proud. Alternativně se u některých systémů využívají nikl-metalhydridové (NiMH) baterie. NiMH baterie jsou odolné vůči hlubokému vybití a mechanickému namáhání. Při vysokém zatížení však poskytují nižší výkon ve srovnání s LiPo bateriemi.\\
Baterie mohou být zapojeny sériově nebo paralelně v závislosti na požadovaném výkonu a kapacitě. Sériové zapojení zvyšuje výsledné napětí při zachování stejné kapacity, což umožňuje vyšší výkon motorů. Paralelní zapojení naopak zvyšuje kapacitu při zachování stejného napětí, což prodlužuje dobu letu dronu.\\
Pro drony jsou ideální baterie s dostatečnou kapacitou (mAh), odpovídajícím počtem článků (S) a vysokým C-ratingem, který udává maximální proudový odběr. Tyto parametry je nutné sladit s výkonovými požadavky dronu. 

\chapter{Fyzika letu dronu}
K zajištění letu kvadrokoptéry je nezbytné, aby byla schopna vykonávat tři základní typy pohybu: vertikální pohyb, laterální pohyb a rotační pohyb. Na základě třetího Newtonova zákona lze každý z těchto pohybů realizovat prostřednictvím čtyř rotorů kvadrokoptéry.

\section{Vertikální pohyb}
Newtonův třetí zákon pohybu stanovuje, že každé akci odpovídá stejně velká, avšak opačně orientovaná reakce.\\
V případě kvadrokoptéry dochází při rotaci jejích rotorů k vytlačování vzduchu směrem dolů, což představuje akční sílu. Na základě uvedeného zákona musí existovat odpovídající reakční síla, která působí směrem vzhůru na kvadrokoptéru. Jakmile velikost této vztlakové síly převýší gravitační sílu působící na kvadrokoptéru, dojde k jejímu vertikálnímu pohybu směrem vzhůru.

\section{Laterální pohyb}
Pokud vztlaková síla působí kolmo vzhůru, kvadrokoptéra se pohybuje vertikálně. Pokud však působí pod úhlem, dochází i k laterálnímu pohybu. Tento jev nastává v důsledku rozkladu vztlakové síly, která působí na dron jak ve vertikálním, tak v horizontálním směru. To způsobuje pohyb kvadrokoptéry do stran nebo ve směru dopředu a dozadu.\\
Laterální pohyb je realizován změnou rychlosti otáčení rotorů. Zvýšení rychlosti dvou rotorů na jedné straně kvadrokoptéry vede k nerovnoměrnému rozložení vztlakové síly. Strana s rychleji rotujícími rotory generuje větší vztlak než opačná strana, což způsobí, že se nakloněná kvadrokoptéra pohybuje směrem k oblasti s nižším vztlakem.

\section{Rotační pohyb}
Posledním typem pohybu kvadrokoptéry je rotace, která je výsledkem působení točivého momentu. Tento moment vzniká v důsledku otáčení rotorů, přičemž podle třetího Newtonova zákona se současně generuje síla stejné velikosti, avšak opačného směru.\\
Točivý moment ovlivňuje kvadrokoptéru při rotaci rotorů, neboť každý jednotlivý rotor generuje svůj vlastní moment síly. Celkový točivý moment působící na kvadrokoptéru je součtem momentů všech čtyř rotorů. Pro eliminaci tohoto momentu se využívá konstrukční řešení, při němž se dva rotory otáčejí ve směru hodinových ručiček a dva v protisměru. Tyto momenty se vzájemně vyruší, čímž se eliminuje nekontrolovaná rotace kvadrokoptéry.\\
Točivý moment lze však také využít k řízené rotaci kvadrokoptéry. Pokud se například rotory 1 a 3 otáčejí rychleji než rotory 2 a 4, výsledný moment síly proti směru hodinových ručiček převáží nad momentem síly ve směru hodinových ručiček, což způsobí rotaci kvadrokoptéry proti směru hodinových ručiček. Naopak, pokud se rotory 2 a 4 otáčejí rychleji než rotory 1 a 3, kvadrokoptéra se začne otáčet ve směru hodinových ručiček.\\
Pro správné fungování je potřeba mít dva druhy vrtulí tak, aby všechny tlačily vzduch směrem k zemi. Jak je patrné na obrázku 9, jeden typ rotoru má levý okraj listu výše v přední části, zatímco druhý typ má výše pravý okraj.
%%%%%%%%%%%%%% PRAKTICKÁ ČÁST %%%%%%%%%%%%%%%%%%	
\part{Konstrukce kvadrokoptéry} % název praktické části (nenechávejte název Praktická část)

\chapter{Součástky}
Zkonstruovaný dron se skládá, bez kterých by jeho realizace byla nemožná. V této kapitole si ukážeme konkrétní použité zařízení a vysvětlíme si jejich specifikace. Tělo dronu se skládá z rámu, motorů, ESC, rozvodné desky napájení a baterie. Flight controller se skládá z Arduina, Bluetooth modulu, modulu gyroskopu-akcelerometru, nepájivého pole. Většina těchto součástek byla zakoupena na online platformě Aliexpress.

\section{Rám}
Při konstrukci dronu byl zvolen 3D tištěný rám pro jeho nízkou hmotnost a dostatečnou mechanickou odolnost. Využití 3D tisku umožnilo výrobu rámu s přesnými rozměry a vlastnostmi dle individuálních požadavků. Pro samotný tisk byl použit standardní materiál PLA, který se vyznačuje snadnou tisknutelností a dostatečnou pevností pro danou aplikaci.\\
Model byl vytvořen jako upravená varianta existujícího designu dostupného na platformě Thingiverse (https://www.thingiverse.com/thing:234867). Tato stránka poskytuje veškeré potřebné modely pro 3D tisk, které jsou volně dostupné ke stažení. Díky této otevřené databázi bylo možné provést úpravy a přizpůsobit konstrukci dronu specifickým požadavkům.\\
Do spojů vytištěného rámu byly integrovány šrouby a matice, které zajišťují pevné spojení jednotlivých komponent. Tato konstrukční metoda zvyšuje mechanickou stabilitu rámu a umožňuje snadnou demontáž či výměnu dílů v případě potřeby.

\section{Motory}
Jako pohonná jednotka byl zvolen bezkartáčový motor 2212 920KV, který nabízí optimální rovnováhu mezi výkonem a spotřebou energie. U dronů se obecně preferují outrunner motory, které jsou díky své konstrukci efektivnější, poskytují vyšší točivý moment a lepší chladicí vlastnosti, což je zásadní pro stabilní a efektivní let.\\
Hodnota KV udává počet otáček motoru za minutu (RPM) na jeden volt napájecího napětí bez zatížení. To znamená, že například motor 2212 920KV při napětí 1V dosáhne přibližně 920 ot./min, při 2V to bude 1840 ot./min atd.\\
Druhé uvedené číslo v označení motoru udává jeho rozměry. První dvě číslice (22) v označení motoru 2212 udávají průměr statoru v milimetrech, zatímco druhé dvě číslice (12) označují výšku statoru v milimetrech.\\
V neposlední řadě je důležité dbát na správnou orientaci závitu tak, aby se utahování vrtule vždy provádělo proti směru otáčení motoru. Toto řešení je klíčové z hlediska bezpečnosti, protože zabraňuje uvolnění vrtule při prudkém zpomalení nebo zastavení motoru. Pro stabilizaci letu je nutné, aby byly použity dva motory s pravotočivou rotací (CW - Clockwise) a dva s levotočivou rotací (CCW - Counterclockwise).\\

\section{Vrtule}
Zvolené vrtule nesou označení 1045, kde první dvě číslice označují průměr vrtule v palcích a poslední dvě číslice představují stoupání vrtule v palcích. Stoupání vrtule (propeller pitch) určuje vzdálenost, kterou by vrtule teoreticky urazila v jednom otočení při ideálních podmínkách. Vyšší stoupání znamená větší objem vzduchu přemístěného směrem dolů během rotace, což zvyšuje tah, ale zároveň zvyšuje aerodynamický odpor a zatížení motoru. Princip lze přirovnat ke závitu šroubu, kde stoupání určuje hloubku zavrtání na jednu otáčku. Obdobně u vrtule odpovídá stoupání tomu, jak hluboko se „zavrtává“ do vzduchu.

\section{ESC}
Brushless motory vyžadují třífázové napájení, což je poměrně složitý proces, který zajišťuje vestavěný mikrořadič v ESC. Z našeho pohledu stačí do ESC přivést napájecí proud a prostřednictvím signálního vodiče mu sdělit, jakou rychlostí se má motor otáčet.\\
Je důležité zajistit, aby ESC bylo kompatibilní s daným bezkartáčovým motorem. Klíčovými parametry jsou maximální napětí a proud, které ESC zvládne. Pro správnou funkci je také nutné ESC spájet s rozvodnou deskou napájení, která rozvádí napětí z baterie ke všem komponentům.

\section{Rozvodná deska napájení}
Rozvodná deska napájení (PDB) slouží k distribuci energie z baterie do všech komponent. Všechny čtyři ESC jsou k této desce připájeny a baterie je k ní připojena přes bezpečný napájecí konektor.

\section{Baterie}
Zdroj veškeré energie pro dron. V této práci byla vybrána baterie se specifikací LiPo 2200mAh 14.8V 30C 4S1P.\\
Začněme částí 4S. Bateriový pack je složen z jednotlivých malých bateriových článků. Jeden LiPo článek má vždy 4,2 V, když je plně nabitý, a 3,7 V, když je vybitý. Označení 4S znamená, že čtyři články jsou zapojeny sériově, což znamená, že výsledné napětí baterie se sčítá. Plně nabité napětí celého bateriového packu tedy činí 16,8 V.\\
Hodnota 2200mAh označuje celkovou kapacitu baterie, což znamená, že při odběru 2,2 A by baterie měla teoreticky vydržet jednu hodinu.\\
Posledním parametrem je C-rating, který udává maximální proud, který je baterie schopna dodat. Maximální proud lze vypočítat jako C-rating × kapacita v Ah:
\begin{center}
    \[
    30C \times 2.2\text{Ah} = 66\text{A}
    \]
\end{center}
Tato hodnota představuje trvalý vybíjecí proud, který může baterie bezpečně poskytovat. Některé baterie mají také burst rating, což znamená, že krátkodobě mohou dodat ještě vyšší proud, například dvojnásobek této hodnoty.

\section{Arduino}
Arduino je programovatelný mikrokontrolér umožňující připojení dalších komponent pro specifické aplikace. Pro tento projekt byla zvolena platforma Arduino UNO, a to především díky vysokému počtu připojovacích pinů a jednoduché manipulaci. Arduino zde plní funkci řídicí jednotky letu (flight controlleru), která na základě vstupních údajů získaných prostřednictvím Bluetooth modulu a akcelerometru provádí výpočty potřebné k regulaci motorů. Zařízení nevyžaduje samostatný napájecí zdroj, neboť je napájeno přímo z motorového obvodu.

\section{Bluetooth modul}
Bluetooth modul HC-05 je bezdrátový komunikační modul umožňující obousměrný přenos dat mezi mikrokontrolérem a externím zařízením. Pro tento projekt byl zvolen právě HC-05 díky své kompatibilitě s platformou Arduino UNO a jednoduché implementaci sériové komunikace prostřednictvím protokolu UART. Modul umožňuje bezdrátové ovládání a přenos řídicích signálů mezi uživatelským rozhraním mobilní aplikace a řídicí jednotkou letu. HC-05 pracuje ve frekvenčním pásmu 2,4 GHz s maximálním dosahem přibližně 10–20 metrů, což je pro tento projekt dostačující. Přenosová rychlost modulu může dosahovat až 1 Mb/s, což umožňuje efektivní a spolehlivou komunikaci mezi zařízením a řídicí jednotkou. 

\section{Akcelerometr a gyroskop}
MPU6050 je měřicí jednotka integrující tříosý gyroskop a tříosý akcelerometr do jednoho čipu, což umožňuje komplexní měření pohybu a orientace v prostoru. Pro tento projekt byl zvolen MPU6050 díky jeho vysoké přesnosti, kompaktním rozměrům a kompatibilitě s platformou Arduino UNO prostřednictvím sběrnice I²C. Modul poskytuje klíčová data o náklonu, zrychlení a úhlové rychlosti, která jsou využívána řídicí jednotkou při výpočtu stabilizačních algoritmů a optimalizaci řízení motorů.

\chapter{Průběh Konstrukce}
Prvním krokem v procesu konstrukce dronu bylo vytvoření jeho rámu pomocí 3D tisku. Po několika neúspěšných pokusech se podařilo úspěšně vytisknout jednotlivé součásti s požadovanou přesností a strukturální pevností. Pro zajištění vyšší soudržnosti konstrukce byly do klíčových částí rámu zataveny kovové závitové vložky, které umožňují pevné uchycení komponent pomocí šroubů. Tyto výztuhy byly umístěny především v centrální části rámu, kde bylo nezbytné pevně připojit ramena ke středu dronu, a na koncích ramen, kde dochází k upevnění motorů s vysokým tahem.\\
Dalším krokem byla instalace a zapojení napájecího systému. Jednotlivé napájecí vodiče motorů byly připájeny k rozvodové desce, která zajišťuje distribuci elektrické energie v celém systému. Stejným způsobem byl k rozvodové desce připojen hlavní napájecí kabel, přes který je k desce připojena baterie. Po dokončení byly motory upevněny na koncích ramen dronu tak, aby jejich směr otáčení odpovídal požadované konfiguraci letového systému.\\
Motory byly ke konstrukci připevněny pomocí šroubků, přičemž pro dodatečnou stabilizaci byly elektronické regulátory otáček (ESC) zajištěny pomocí instalačních pásků.\\
Po dokončení mechanické konstrukce dronu následoval klíčový krok, kterým je návrh a implementace řídicí jednotky letu (flight controlleru).

\chapter[Vývoj flight controlleru]{Vývoj flight controlleru}
%%% v obsahu se objeví jen to co je v hranatých závorkách
Hlavní řídicí jednotkou flight controlleru je Arduino UNO. Prvním krokem bylo připojení Bluetooth modulu HC-05 a modulu MPU6050, který integruje akcelerometr a gyroskop. Tyto komponenty byly propojeny s Arduinem pomocí vodivých vodičů a nepájivého kontaktního pole, což umožňuje flexibilní zapojení a případné úpravy obvodu.\\
Samotný řídicí software pro Arduino byl vytvořen v prostředí Arduino IDE, které je speciálně uzpůsobeno pro vývoj aplikací na platformě Arduino. Programovací jazyk použitý v kódu je C++, což umožňuje efektivní práci s hardwarem a implementaci nízkoúrovňových operací nezbytných pro řízení dronu.\\
Kromě programování řídicí jednotky bylo nutné vyvinout také mobilní aplikaci, která umožňuje ovládání dronu prostřednictvím Bluetooth komunikace. Pro tento účel byla využita aplikace Bluetooth Electronics, která poskytuje širokou škálu přizpůsobitelných tlačítek a předpřipravených ovládacích panelů. Tato aplikace rovněž umožňuje nastavení jednotlivých příkazů odesílaných přes Bluetooth a automaticky zajišťuje propojení mezi mobilním zařízením a Bluetooth modulem dronu.\\
Profesionální drony jsou vybaveny pokročilým softwarem pro stabilizaci a automatickou korekci polohy. Konstrukčně realizovaný dron však funguje pouze na principu přímého příjmu ovládacích příkazů, přičemž neobsahuje stabilizační algoritmy. Výsledkem je vysoká nestabilita a značná náchylnost k vychýlení během letu. Implementace stabilizačního systému by vyžadovala složité programování, které by významně rozšířilo rozsah této maturitní práce.\\
Navržený dron je kompletně zhotoven po hardwarové stránce, přičemž zbývající vývoj pro stabilizovaný let se týká implementace softwarového stabilizačního systému v Arduinu.



\chapter{priklady v latexu}
text
Odkaz v závorkách: \parencite[see][page 900]{einstein}\\
Odkaz: \cite{knuthwebsite}\\
A odkaz pod čarou: \footcite[see][s. 42]{latexcompanion}\\
Dobrý den, ahoj, \gls{atd}\\
Praha, \gls{tj} hlavní město ČR
text

\begin{figure}
  \includegraphics[width=\linewidth]{test.jpg}
  \caption{Testovací}
  \label{fig:test}
\end{figure}
Obrázek \ref{fig:test} ukazuje Shangai z Pixabay.\\
Tabulka \ref{tab:test2} ukazuje hádejte, co.
	
\lipsum[3]


Výpis programu \nameref{lst:hello_world}  naleznete ve výpise \ref{lst:hello_world}.

\begin{lstlisting}[title={Program hello.c}, caption={hello.c}, label={lst:hello_world}]
#include <stdio.h>
#define CISLO 10

int main(void) {
	int i = CISLO;

	print("Hello World!\n");
	print("%d", i);

	return (0);
}
\end{lstlisting}

\lipsum[1]	

\begin{lstlisting}[numbers=none, title={Příklad výstupního souboru}]
11.0524
5.5954
6.7996
13.8584
15.1357
Soucet: 52.4415
\end{lstlisting}

\chapter{Program}
Program Arduina je navržen tak, aby umožňoval bezdrátové řízení kvadrokoptéry prostřednictvím mobilní aplikace Bluetooth Electronics, která využívá Bluetooth komunikace. Program přijímá vysílané příkazy, analyzuje je a na jejich základě ovládá jednotlivé motory dronu. Každý ovládací prvek v aplikaci vysílá specifický signál s předponou, která umožňuje jeho správnou interpretaci.\\
Ovládání kvadrokoptéry je realizováno několika prvky. Směrová tlačítka umístěná na levé straně ovladače ve tvaru „+“, vysílají příkazy s předponou p: a číslem 1 až 4. Tlačítko šipky nahoru slouží ke zvýšení výkonu motorů, šipka dolů jej naopak snižuje. Směrová tlačítka vlevo a vpravo ovlivňují rotaci dronu kolem jeho vertikální osy. Při uvolnění jakéhokoliv tlačítka se odešle příkaz p:0.\\
K řízení směru letu slouží pravý joystick, jehož výchylka určuje náklon dronu. Tento prvek využívá příkazy s předponou m: a souřadnicemi následující písmena X a Y, které odpovídají aktuální poloze joysticku na obrazovce mobilního zařízení. Na základě míry vychýlení joysticku se kvadrokoptéra nakloní v příslušném směru, čímž dochází ke změně trajektorie letu.\\
Důležitou součástí ovládacího systému jsou také bezpečnostní prvky. Červené tlačítko v horní části ovladače slouží jako nouzový vypínač a při jeho stisknutí dojde k okamžitému zastavení všech motorů. Žluté tlačítko zatím nemá přiřazenou funkci.\\
Pro zajištění spolehlivého provozu obsahuje program textový výstup, který umožňuje monitorování přijatých příkazů a jejich zpracování. Tyto informace lze zobrazit při připojení Arduina k počítači přes USB kabel. Program rovněž implementuje bezpečnostní mechanismus, který zajistí, že v případě ztráty spojení mezi mobilním zařízením a kvadrokoptérou dojde k okamžitému zastavení motorů.\\
Při nahrávání nového kódu do Arduina je nezbytné dočasně odpojit vodiče připojené k pinům 0 a 1, jelikož tyto piny jsou využívány během inicializace programu. Po úspěšném nahrání je nutné vodiče znovu připojit na původní místo, aby mohla kvadrokoptéra správně komunikovat s mobilní aplikací. Dalším bezpečnostním opatřením je zabudovaná ochrana motorů, která vyžaduje, aby byl jejich výkon při startu nastaven na nulu. Teprve poté mohou být motory aktivovány.\\
Tento program tak umožňuje efektivní a bezpečné ovládání kvadrokoptéry, přičemž zohledňuje jak uživatelskou přívětivost, tak i bezpečnostní aspekty provozu.

\chapter{Schéma zapojení}
Propojení Arduina s Bluetooth modulem, akcelerometrem, gyroskopem, baterií, regulátory ESC a motory je znázorněné na obrázku v programu Fritzing. Na schématu je patrný způsob napájení. Celá pohonná soustava čerpá energii přímo z připojené baterie, zatímco samotné Arduino je napájeno přes ESC propojené s 5V pinem. Dva z bezkartáčových motorů mají odlišné zapojení vodičů, což způsobuje jejich opačnou rotaci.

\vspace{40pt}
\begin{figure}[H]
	\centering
	\includegraphics[width=\linewidth]{drone-scheme.png}
	\caption{Obrázek x.y: Schéma propojení modulů s Arduinem a pohonného systému}
	\label{fig:drone-scheme.png}
  \end{figure}

%%%%%%%%%%%%% ZÁVĚR
\chapter*{Závěr}
	
\lipsum[1]
	
\nocite{*}
\printbibliography					% Vytvoří seznam literatury
\addcontentsline{toc}{chapter}{Bibliografie}
\printglossary[title={Zkratky}]		% Vytvoří seznam zkratek
\listoffigures						% Vytvoří seznam obrázků

%%%%%%%%%%%%% PŘÍLOHY - APPENDIX 	
\begin{appendices}
	\chapter{Fotografie zkonstruované kvadrokoptéry}	
	\lipsum[1]
    	%\pitem{Fotky z pokusů}
    	%\eitem{Vlastní program}
    	%\eitem{Dokumentace}
    	%\eitem{Testovací data}
\chapter{Kód programu Arduina}
\begin{lstlisting}[title={}, caption={}, label={}, basicstyle=\footnotesize\ttfamily, inputencoding=utf8]
#include <Servo.h> // Knihovna pro ovladani ESC pomoci PWM signalu

String data = ""; // Retezec pro ukladani prijatych dat pres Bluetooth
Servo ESC1;     // Vytvoreni objektu servo pro rizeni ESC motoru 1
Servo ESC2;
Servo ESC3;
Servo ESC4;
int power;      // Promenna pro ukladani vykonu motoru (0-100 %)
int escValue;   // Promenna pro prepocet vykonu na hodnotu pro ESC (0-180)
const int bluActive = 2; // Pin pro kontrolu stavu pripojeni Bluetooth modulu HC-05 (STATE pin)

void processCommand() {
// Zpracovani prikazu zacinajicich "p:"
if (data.startsWith("p:")) {
	String value = data.substring(2); // Extrahovani hodnoty po "p:"
	int number = value.toInt();       // Prevedeni na cele cislo

	Serial.print("Zpracovavam prikaz: p:");
	Serial.println(number);

	// Zvyseni vykonu motoru
	if (number == 1) {
	if (power <= 95) {
		power += 5;
		Serial.print("Vykon: ");
		Serial.println(power);
	} else {
		power = 100;
	}

	// Rotace doleva p:2 (zatim neimplementovano)
	} else if (number == 2) {
	Serial.println("Akce pro p:2");

	// Snizeni vykonu motoru
	} else if (number == 3) {
	if (power >= 5) {
		power -= 5;
		Serial.print("Vykon: ");
		Serial.println(power);
	} else {
		power = 0;
	}

	// Rotace doprava p:4 (zatim neimplementovano)
	} else if (number == 4) {
	Serial.println("Akce pro p:4");

	} else {
	Serial.println("Neznamy prikaz po p:");
	}
}

// Zpracovani prikazu zacinajicich "b:"
else if (data.startsWith("b:")) {
	String value = data.substring(2); // Extrahovani hodnoty po "b:"
	int number = value.toInt();       // Prevedeni na cele cislo

	Serial.print("Zpracovavam prikaz: b:");
	Serial.println(number);

	// Zastaveni vsech motoru
	if (number == 0) {
	Serial.println("Tlacitko b0 stisknuto");
	power = 0;
	// Neimplementovana funkcionalita druheho tlacitka
	} else if (number == 1) {
	Serial.println("Tlacitko b1 stisknuto");
	} else {
	Serial.println("Neznamy prikaz b:");
	}
}

// Zpracovani prikazu zacinajicich "m:"
// Naklon dopredu, do zadu a dostran (zatim neimplementovano)
else if (data.startsWith("m:")) {
	int xIndex = data.indexOf('X'); // Najiti pozice 'X'
	int yIndex = data.indexOf('Y'); // Najiti pozice 'Y'

	if (xIndex != -1 && yIndex != -1) { // Overeni, ze obe hodnoty existuji
	String xValue = data.substring(xIndex + 1, yIndex); // Extrahovani hodnoty X
	String yValue = data.substring(yIndex + 1);         // Extrahovani hodnoty Y

	int x = xValue.toInt(); // Prevedeni hodnoty X na cele cislo
	int y = yValue.toInt(); // Prevedeni hodnoty Y na cele cislo

	Serial.print("X: ");
	Serial.println(x);
	Serial.print("Y: ");
	Serial.println(y);

	Serial.println("Zpracovavam prikaz m:...");
	} else {
	Serial.println("Neplatny format prikazu m:");
	}
}

// Ignorovani zprav, ktere nezacinaji "p:", "b:", nebo "m:"
else {
	Serial.println("Zprava ignorovana (neznamy format)");
}

// Vymazani datoveho retezce pro dalsi zpravy
data = "";
}

void setup() {
Serial.begin(9600); // Spusteni seriove komunikace
Serial.println("Pripraveno na prijem dat pres Bluetooth...");

// Pripojeni ESC motoru na piny 4-7
ESC1.attach(4, 1000, 2000); // (pin, min sirka pulzu, max sirka pulzu v mikrosekundach)
ESC2.attach(5, 1000, 2000);
ESC3.attach(6, 1000, 2000);
ESC4.attach(7, 1000, 2000);

power = 0; // Vychozi vykon motoru
pinMode(bluActive, INPUT); // Nastaveni pinu pro cteni stavu Bluetooth pripojeni
}

void loop() {
// Kontrola, zda je k dispozici data ze seriove komunikace
if (Serial.available() > 0) {
	char incomingChar = Serial.read(); // Precteni znaku z Bluetooth

	if (incomingChar == '\n') { // Kontrola konce zpravy (novy radek)
	Serial.print("Prijata zprava: ");
	Serial.println(data); // Vypsani prijate zpravy
	processCommand(); // Zpracovani prijateho prikazu
	} else {
	data += incomingChar; // Pridani znaku do retezce zpravy
	}
}

// Kontrola stavu Bluetooth pripojeni
int connectionStatus = digitalRead(bluActive); // Cteni pinu STATE z HC-05
if (connectionStatus == LOW) { // Pokud je stav LOW, Bluetooth je odpojeno
	power = 0; // Reset vykonu motoru na vychozi hodnotu
	Serial.println("Bluetooth odpojeno!");
}

// Prevod vykonu (0-100) na hodnotu (0-180) pro ESC motory
escValue = map(power, 0, 100, 0, 180); 
ESC1.write(escValue); // Nastaveni vykonu pro motor 1
ESC2.write(escValue);
ESC3.write(escValue);
ESC4.write(escValue);
}	
\end{lstlisting}  
\end{appendices}
%%%%%%%%%%%%%%%
\end{document}
%%%%%%%%%%%%%%%%%%%% KONEC %%%%%%%%%%%%%%%%%%%%%%%%%